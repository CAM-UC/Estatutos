\documentclass[letterpaper,11pt]{article}
% Packages
\usepackage[spanish]{babel}
\usepackage[utf8]{inputenc}
\usepackage[usenames,dvipsnames]{color}
\usepackage[margin=1in]{geometry}
\usepackage[T1]{fontenc}
\usepackage{lmodern}
\usepackage{textcomp}
\usepackage{hyperref}
\usepackage{fancyhdr}
\usepackage{titlesec}
\usepackage{enumitem}
\usepackage{tocloft}
\usepackage{fmtcount}
\usepackage{etoolbox}

\definecolor{blue(pigment)}{rgb}{0.2,0.2,0.7}

% Formato
\hypersetup{colorlinks= true, linkcolor=blue(pigment), urlcolor=blue(pigment)}
\spanishdecimal{.}
\tolerance=1000
\hyphenpenalty=1000
\setlength{\parindent}{0in}
\setlength{\parskip}{0.1in}
\setlength\textheight{8.2in}
\setlength\topmargin{-1in}
\lhead{\sc Estatutos Centro de Alumnos de Matemáticas\\Pontificia Universidad Católica de Chile}
\chead{}
\rhead{}
\lfoot{}
\cfoot{\thepage}
\rfoot{}
\fancypagestyle{plain}{%
\fancyhf{}
\cfoot{\thepage}
\lfoot{}
\renewcommand{\headrulewidth}{0pt}}
\renewcommand{\headrulewidth}{1 pt}
\renewcommand{\footrulewidth}{0 pt}
\headheight=75pt
\setlength{\cftbeforesecskip}{12pt}
\setlength{\cftbeforesubsecskip}{6pt}
\cftsetindents{section}{0em}{1.75em}

% Ambientes de Artículos
\newcounter{art}
\newenvironment{art}{\refstepcounter{art}\mbox{\textbf{Art.{\space}\theart.}}\ignorespaces}{}
\newcounter{artt}
\newenvironment{artt}{\refstepcounter{artt}{\bf\Ordinalstring{artt}[f].}}{} % TODO Check spacing?
% Enumeraciones
\renewcommand{\theenumi}{\roman{enumi}}
\renewcommand{\labelenumi}{\theenumi.}
\renewcommand{\theenumii}{\arabic{enumii}}
\renewcommand{\labelenumii}{(\theenumii)}
\renewcommand{\theenumiii}{\roman{enumiii}}
\renewcommand{\labelenumiii}{\theenumiii.}
\renewcommand{\theenumiv}{(\alph{enumiv})}
\newcommand{\HRule}{\rule{\linewidth}{0.5mm}}
%\newcommand{\aref}[1]{\hyperref[#1]{\ref*{#1}}}
%\newcommand{\aaref}[2]{\hyperref[#2]{\ref*{#1}, letra \ref*{#2}}}
\makeatletter \renewcommand\p@enumii{\theenumi, } \makeatother
\makeatletter \renewcommand\p@enumiii{\theenumii, } \makeatother
\makeatletter \renewcommand\p@enumiv{\theenumiii, } \makeatother

% Formalidades respecto al título, autor y fecha de modificación
\title{Estatutos}
\author{Centro de Alumnos de Matemáticas}
\date{Marzo 2020}

% Inicio del documento
\begin{document}
\pagenumbering{gobble}
\thispagestyle{plain}
\vspace*{-75pt}

\begin{center}
    \begin{Large}
        {\bf
            ESTATUTOS DEL CENTRO DE ALUMNOS DE MATEMÁTICAS

            PONTIFICIA UNIVERSIDAD CATÓLICA DE CHILE
        }
    \end{Large}

    \vspace*{30pt}

\end{center}
\tableofcontents
\newpage
\pagenumbering{arabic}

\section{Declaración de Principios}\label{principios}
``Nosotros, la comunidad de estudiantes de la Facultad de Matemáticas ratificamos nuestro compromiso con la defensa de los derechos e intereses de todos los estudiantes de la Facultad de Matemáticas, siendo rasgos fundamentales del Centro de Alumnos de Matemáticas su naturaleza participativa, pluralista y el respeto a las decisiones de las mayorías. Junto con ello, manifestamos nuestro respeto a los valores católicos.''

\section{Normas Generales}\label{normasGenerales}
\begin{art}\label{}
    Se establecen como normas estatutarias del Centro de Alumnos de Matemáticas (desde ahora ``CAM'') las contenidas en los presentes estatutos.
\end{art}

\begin{art}\label{estudiantesRepresentados}
    El CAM es el organismo que representa ante la comunidad universitaria de la Pontificia Universidad Católica de Chile (UC), dentro de su ámbito de competencia, a los ``estudiantes representados'', quienes están descritos a continuación:
    \begin{enumerate}
        \item Estudiantes de pregrado regulares de la Facultad de Matemáticas (desde ahora ``la Facultad'').
        \item Estudiantes de postgrado regulares de la Facultad que manifiesten su deseo de ser representados ante el TRICEL\footnote{Como se define en la sección \ref{TRICEL}}.
    \end{enumerate}
\end{art}

\begin{art}\label{finalidadesCAM}
    El CAM tiene por finalidades:
    \begin{enumerate}
        \item Velar por los derechos e intereses de los estudiantes representados.
        \item Promover, impulsar y desarrollar las actividades conducentes a la correcta formación, integración y recreación de los estudiantes representados.
        \item Procurar que la enseñanza de la Facultad sea de excelencia, creando profesionales comprometidos con la sociedad.
        \item Administrar los fondos puestos a disposición del CAM por su antecesor, por la Facultad y por la Federación de Estudiantes de la UC (desde ahora ``FEUC''). Asimismo, administrar los recursos que se generen a través de otro tipo de instancias, como actividades recaudadoras de fondos o donaciones.
    \end{enumerate}
\end{art}

\begin{art}\label{organismosDirectivos}
    Los organismos directivos del CAM son los siguientes:
    \begin{enumerate}
        \item Directiva: Está conformada por cinco cargos. En orden jerárquico:
              \begin{enumerate}
                  \item Presidencia.
                  \item Vicepresidencia Interna.
                  \item Vicepresidencia Externa.
                  \item Secretaría.
                  \item Tesorería.
              \end{enumerate}
        \item Consejo Ejecutivo: Está conformado por miembros designados por quien ocupe el cargo de la Presidencia según lo que estime conveniente, especificando su labor, previa aprobación individual del Consejo Estudiantil.
        \item Consejo de Delegados: Se conformará por los delegados de cada generación de cada carrera, un delegado de postgrado y un delegado correspondiente a todos los estudiantes no considerados en los grupos anteriores independiente de su carrera. Hay cuatro de Matemáticas, cuatro de Ciencias de Datos y cinco generaciones de Estadística, las cuales se dividen como se ve en la tabla.
              \begin{figure*}[h]
                  \begin{tabular}{|c|c|c|c|}
                      \hline
                      Año ingreso   & Generación Matemáticas & Generación Ciencias de Datos & Generación Estadística \\
                      \hline
                      Año actual    & Generación 1           & Generación 1                 & Generación 1           \\
                      Año actual -1 & Generación 2           & Generación 2                 & Generación 2           \\
                      Año actual -2 & Generación 3           & Generación 3                 & Generación 3           \\
                      Año actual -3 & Generación 4           & Generación 4                 & Generación 4           \\
                      Año actual -4 &                        &                              & Generación 5           \\
                      \hline
                  \end{tabular}
              \end{figure*}
        \item Consejerías Académicas: Hay tres consejerías académicas, cada una corresponde a una carrera de pregrado, conformada respectivamente por un Consejero y además por un Subconsejero si así el Consejero lo desea.
        \item Consejo Estudiantil: Está conformado por el Consejo de Delegados, las Consejerías Académicas, el Consejero Territorial y dos miembros de la Directiva. Los Consejos que se convoquen serán abiertos y podrá participar con derecho a voz todo estudiante representado que así lo desee.
    \end{enumerate}
\end{art}

\begin{art}\label{organismosNoDirectivos}
    El Tribunal Calificador de Elecciones\footnote{Como se define en la sección \ref{TRICEL}.} (TRICEL) es un organismo no directivo del CAM.
\end{art}

\begin{art}\label{}
    Los estudiantes representados que estén en el proceso de cambio de carrera serán representados por el CAM hasta que se haga efectivo. Además, al hacerse efectivo el cambio, el estudiante deberá informarlo a los respectivos centros de estudiantes.
\end{art}

\begin{art}\label{}
    Si un estudiante representado está cursando más de una carrera, y una de ellas no pertenece a la Facultad, se verá obligado a informar a los centros de estudiantes correspondientes y al TRICEL bajo qué carrera votará en votaciones que conciernan a la FEUC.
\end{art}

\section{De las Atribuciones, Derechos y Deberes}\label{atribucionesDerechosDeberes}

\begin{art}\label{derechosEstudiantes}
    Son derechos de los estudiantes representados:
    \begin{enumerate}
        \item Manifestar libremente su opinión en las Asambleas Generales\footnote{Como se definen en el artículo \ref{asambleas}} y otros espacios destinados por el CAM para tal efecto.
        \item Publicar información y/u opiniones en espacios destinados por el CAM, siempre y cuando el CAM no ponga objeciones.
        \item Participar en las actividades que organice el CAM.
        \item Exigir información sobre el trabajo del CAM en forma oral o escrita.
        \item Proponer proyectos al CAM, a través del Consejo Estudiantil, como se menciona en el artículo \ref{funcionesConsejoEstudiantil}.
        \item Solicitar espacios, equipos y/o materiales que sean del CAM para realizar actividades que brinden beneficios a la comunidad universitaria.
        \item Consultar sobre su situación académica a la Consejería Académica correspondiente.
    \end{enumerate}
\end{art}

\begin{art}\label{deberesCAM}
    Son deberes de los miembros de los organismos directivos del CAM:
    \begin{enumerate}
        \item Informar a cada uno de los estudiantes de sus derechos.
        \item Votar en las elecciones a nivel CAM.
        \item Justificar por escrito inasistencias a las Asambleas Generales. Esta justificación quedará registrada en el acta de la Asamblea correspondiente.
        \item Respetar la opinión y los espacios de democracia que se generen.
        \item Velar por el cumplimiento de los estatutos.
    \end{enumerate}
\end{art}

\begin{art}\label{deberesDirectiva}
    Son deberes de la Directiva:
    \begin{enumerate}
        \item Desarrollar el plan de trabajo presentado en la medida de lo posible.
        \item Convocar a las Asambleas Ordinarias a las que hace referencia la sección \ref{asambleasConsejos}.
        \item Crear y promover el desarrollo de actividades de orden estudiantil y/o de formación general e integral en el estudiantado.
        \item Apoyar a las vocalías\footnote{Como se definen en la sección \ref{vocalias}}.
        \item Pronunciarse, tras debida solicitud por parte de los estudiantes representados, ante los problemas estudiantiles y sociales, e informar su postura al respecto.
        \item Representar ante autoridades universitarias, organismos superiores y la FEUC, a los estudiantes representados.
        \item Administrar los bienes que estén a disposición del CAM.
        \item Informar a los estudiantes representados de las actividades realizadas y a realizar de su posible interés.
        \item Votar en todas las votaciones de toda Asamblea General en la que este presente.
        \item Asistir con la presencia de al menos uno de sus miembros a los Consejos de Federación, de Facultad y Estudiantil según corresponda.
        \item Tomar decisiones con total acuerdo de la Directiva, o como sus miembros lo estimen conveniente.
        \item Definir los medios oficiales de información, los cuales deben incluir al correo institucional.
        \item Todas las demás disposiciones que el presente estatuto estipule.
    \end{enumerate}
\end{art}

\begin{art}\label{atribucionesDirectiva}
    Son atribuciones de la Directiva:
    \begin{enumerate}
        \item Convocar a sesiones extraordinarias de Asambleas Generales\footnote{Como se describe en el artículo \ref{asambleas}}.
        \item Sesionar cuantas veces estime conveniente para cumplir con las funciones que contemple el actual estatuto.
        \item Pronunciarse, según lo estimen conveniente, ante los problemas estudiantiles y sociales, e informar su postura al respecto.
        \item Todas las demás disposiciones que el presente estatuto estipule.
    \end{enumerate}
\end{art}

\begin{art}\label{funcionesPresidencia}
    Son funciones de quien esté a cargo de la Presidencia:
    \begin{enumerate}
        \item Representar a la Directiva tanto dentro como fuera de la universidad, de acuerdo a las finalidades de la Directiva.
        \item Garantizar la existencia de un vocero de las inquietudes de los estudiantes.
        \item Asegurar el adecuado desempeño de las Asambleas Generales.
        \item Presidir reuniones de la Directiva.
        \item Realizar cuentas públicas como se describe en el artículo \ref{asambleas}, donde se deberá entregar una evaluación con los hitos más importantes del semestre, junto con un estado financiero que refleje el gasto de dinero del CAM.
        \item Ampliar el número del Consejo Ejecutivo, sus funciones y atribuciones dentro de lo que estime conveniente, con previa aprobación del Consejo Estudiantil.
    \end{enumerate}
\end{art}

\begin{art}\label{funcionesVicepresidenciaInterno}
    Son funciones de quien esté a cargo de la Vicepresidencia Interna:
    \begin{enumerate}
        \item Apoyar a la Presidencia en las funciones que este deba cumplir con la comunidad de la Facultad.
        \item Subrogar a quien tenga el cargo de la Presidencia en su ausencia.
        \item Velar por el cumplimiento de las funciones de los distintos miembros de la Directiva y de los miembros del Consejo Estudiantil.
        \item Mediar las relaciones entre la Directiva y las autoridades, profesores y funcionarios de la Facultad. Asimismo, deberá coordinar cualquier trabajo con el Consejo Estudiantil.
        \item Difundir toda la información que la Directiva deba y/o quiera por todos los medios oficiales.
    \end{enumerate}
\end{art}

\begin{art}\label{funcionesVicepresidenciaExterno}
    Son funciones de quien esté a cargo de la Vicepresidencia Externa:
    \begin{enumerate}
        \item Apoyar a la Presidencia en las funciones que este deba cumplir con la comunidad universitaria externa a la Facultad.
        \item Subrogar a quien tenga el cargo de la Vicepresidencia Interna en su ausencia.
        \item Mediar las relaciones entre la Directiva y las autoridades universitarias, la FEUC, movimientos políticos, las instituciones y los organismos no pertenecientes a la universidad.
        \item Asistir al Consejo FEUC. En caso de no poder hacerlo, coordinar un reemplazo con la Directiva y el Consejo Ejecutivo.
        \item Informar e invitar al Consejero Territorial a los consejos estudiantiles cuando estos se realicen.
    \end{enumerate}
\end{art}

\begin{art}\label{funcionesSecretaria}
    Son funciones de quien esté a cargo de la Secretaría:
    \begin{enumerate}
        \item Publicar en todos los medios oficiales las citaciones a Asambleas Generales\footnote{Como se definen en artículo \ref{asambleas}}, y después publicar las actas por los mismos medios.
        \item Mantener el inventario al día y tenerlo a disposición de la Directiva cuando esta lo estime conveniente.
        \item A comienzos de año dar a conocer los estatutos que rigen al CAM, publicándolos en todos los medios oficiales.
        \item Recibir proyectos de reforma a los estatutos\footnote{Ver sección \ref{estatutos}}.
        \item Tomar acta de las Asambleas y reuniones de Directiva tanto ordinarias como extraordinarias.
    \end{enumerate}
\end{art}

\begin{art}\label{funcionesTesoreria}
    Son funciones de quien esté a cargo de la Tesorería:
    \begin{enumerate}
        \item Llevar la contabilidad del CAM, entendiéndose como mantener el libro de ingresos y egresos al día, mantener todas la boletas y todo lo que compruebe las cuentas realizadas. Tener la contabilidad a disposición de todo estudiante representado.
        \item Presentar un informe trimestral de haberes.
        \item Administrar la cuenta de ahorros y/o corriente del CAM, si existiere.
        \item Administrar los fondos de becas del CAM.
        \item Evaluar y administrar los gastos y ganancias que impliquen los proyectos que se generen por parte de la comunidad.
    \end{enumerate}
\end{art}

\begin{art}\label{funcionesDelegados}
    Son funciones de los Delegados:
    \begin{enumerate}
        \item Ayudar a la Directiva en las actividades que se realicen para la comunidad, tanto en la realización como en su difusión.
        \item Difundir información que el CAM entregue al nivel que pertenezca.
        \item Asistir a los Consejos Estudiantiles que se convoquen.
        \item Difundir, con debida anticipación, las convocatorias de los Consejos Estudiantiles a los estudiantes que representa.
    \end{enumerate}
\end{art}

\begin{art}\label{atribucionesDelegados}
    Es atribución de los Delegados proponer y llevar a cabo proyectos aprobados por el Consejo Estudiantil.
\end{art}

\begin{art}\label{deberesConsejeriaAcademica}
    Son deberes de la Consejería Académica:
    \begin{enumerate}
        \item Asistir a los comités curriculares de la Facultad, según lo establecido en la Normativa de Comités Curriculares de la Vicerrectoría Académica.
        \item Asistir a los Consejos de Facultad.
        \item Estudiar y desarrollar proyectos y propuestas académicas para los estudiantes representados. También organizar actividades dirigidas al bienestar de los mismos, y mantener constante contacto con ellos.
        \item Estar al tanto e informar a los estudiantes representados sobre el proceso de permanencia.
        \item Asistir a los estudiantes representados en problemas de índole académica, como evaluaciones o cargas académicas, llevando a la correcta gestión de estos.
    \end{enumerate}
\end{art}

\begin{art}\label{funcionesConsejeroTerritorial}
    Son funciones y atribuciones del Consejero Territorial que represente a la Facultad de Matemática todos aquellos dispuestos en los estatutos FEUC, y aquellos dispuestos en este estatuto.
\end{art}

\begin{art}\label{funcionesConsejoEstudiantil}
    Son funciones y atribuciones del Consejo Estudiantil:
    \begin{enumerate}
        \item Elegir un Director de Consejo entre sus integrantes, quien moderará sus sesiones.
        \item Elegir un Secretario entre sus integrantes, quien llevará las actas del Consejo para su posterior registro. Las actas quedarán a libre disposición de todo estudiante representado.
        \item Velar por el buen trabajo del CAM.
        \item Proponer, modificar y aprobar proyectos.
        \item Votar sobre proyectos que el CAM proponga.
        \item En caso de que el Consejo lo estime necesario, destituir a cualquier miembro de un organismo directivo del CAM. Esto se regirá según lo descrito en el artículo \ref{perdidaCargosCAM} y el artículo \ref{destitucionesConsejo} de la sección \ref{vacancias}.
    \end{enumerate}
\end{art}

\section{De las Asambleas y Consejos}\label{asambleasConsejos}
\begin{art}\label{asambleas}
    Las Asambleas tienen carácter informativo y/o consultivo, pudiendo transformarse en resolutivas si es que hay un quórum mínimo de 40\% de los estudiantes representados y es aprobado por la mayoría simple de los presentes en la Asamblea. Si esto pasara, se requerirá de al menos un miembro del TRICEL calificando las votaciones realizadas en Asamblea. Hay dos tipos de Asambleas:
    \begin{enumerate}
        \item Asamblea Ordinaria: Debe ser convocada al menos dos veces por semestre, tomando lugar, respectivamente, durante los primeros y últimos veinte días hábiles de aquél. Cada Asamblea Ordinaria deberá ser convocada con al menos tres días hábiles de anticipación. En cada una de ellas se deberá hacer una cuenta pública.
        \item Asamblea Extraordinaria: Convocada por la Directiva, o por un grupo de mínimo 7 estudiantes representados que previamente haya entregado sus firmas a la Directiva. Esta deberá ser citada con al menos dos días hábiles de anticipación.
    \end{enumerate}
\end{art}

\begin{art}\label{consejos}
    Los Consejos Estudiantiles son de dos tipos:
    \begin{enumerate}
        \item Consejo Ordinario: Es convocado por el Director del Consejo como mínimo una vez cada mes, exceptuando los meses de enero, febrero y julio, y con al menos dos días hábiles de anticipación. Cualquier decisión resolutiva requiere un mínimo del 50\% de la asistencia del total de los integrantes del Consejo Estudiantil.
        \item Consejo Extraordinario: Puede ser convocado por cualquier miembro del CAM, con al menos tres horas de anticipación. Cualquier decisión resolutiva require un mínimo del 60\% de la asistencia del total de los integrantes del Consejo Estudiantil.
    \end{enumerate}
\end{art}

\begin{art}\label{}
    Cuando una sesión del Consejo Estudiantil es convocada y no está presente su Director o su Secretario se designará a un Director o un Secretario interino, según corresponda, que ejercerá tal cargo durante la misma sesión.
\end{art}

\section{De la elección de cargos}\label{elecciones}

\begin{art}\label{eleccionesRequerimientosPostulación}
    Para postular a un cargo del CAM, se requiere cumplir con las siguientes condiciones:
    \begin{enumerate}
        \item Ser estudiante representado.
        \item Ser estudiante de pregrado en la Facultad, con excepción del delegado de postgrado.
        \item Tener un Promedio General Acumulado (PGA) de al menos 4,0.
        \item Tener menos de dos alertas académicas (sistema de permanencia).
    \end{enumerate}
\end{art}


\begin{art}\label{eleccionesConvocación}
    Será responsabilidad de la Directiva en ejercicio convocar a las elecciones correspondientes.
\end{art}

\subsection{De la elección de la Directiva}\label{eleccionesCAM}
\begin{art}\label{}
    La Directiva CAM se elegirá anualmente por medio de una votación como se describe en la sección \ref{votaciones}.
\end{art}

\begin{art}\label{eleccionesDirectivaPostulacion}
    Cada lista postulante tendrá que designar a un integrante por cada cargo de la Directiva, teniendo la opción de postular su respectivo Consejo Ejecutivo. Además, para formalizar la inscripción se debe hacer llegar la lista con todos los integrantes a la Directiva en ejercicio.
\end{art}

\begin{art}\label{eleccionesDirectivaFechas}
    El proceso de elecciones comenzará con anuncio electoral por parte de la Directiva que, con una antelación mínima de cinco días hábiles, fijará el inicio de los plazos de inscripción, que tendrán lugar durante la segunda semana de octubre. Las elecciones se realizarán entre cinco y ocho días hábiles tras el fin de la inscripción y estarán bajo la supervisión del TRICEL según lo disponen los presentes estatutos.
\end{art}

\begin{art}\label{eleccionesDirectivaPublicacion}
    Al término del proceso de inscripción, Secretaría deberá publicar en los medios oficiales las listas de candidatos y sus respectivos programas.
\end{art}

\begin{art}\label{eleccionesDirectivaInscripcion}
    Si una vez cumplido el plazo de inscripción no hay postulaciones, el Consejo Estudiantil decidirá si agregará un plazo extraordinario de a lo más cinco días hábiles o designar una Directiva interina compuesta por miembros del Consejo Estudiantil.
\end{art}

\begin{art}\label{eleccionesDirectivaFalla}
    En caso de no tener postulaciones o de ser rechazadas las listas, se designará una Directiva interina tal como lo menciona el artículo \ref{eleccionesDirectivaInscripcion}, ejerciendo en sus cargos hasta una nueva convocatoria, que se realizará el siguiente año junto con la elección de delegados.
\end{art}

\begin{art}\label{eleccionesDirectivaVotacion}
    Para que la elección tenga validez se debe cumplir con los puntos descritos en la sección \ref{votaciones}.
\end{art}

\subsection{De la elección de la Consejería Académica}\label{eleccionesConsejeria}

\begin{art}\label{eleccionesConsejeriaLista}
    Cada postulación a la Consejería Académica será conformada por un Consejero Académico.
\end{art}

\begin{art}\label{eleccionesConsejeriaFalla}
    Si al final del plazo de inscripción no hay postulaciones, se agregará un plazo extraordinario de a lo más cinco días hábiles. En caso de no haber una postulación en este plazo extraordinario, o que se rechacen las postulaciones, se considerará el puesto vacante y se seguirá lo descrito en el artículo \ref{vacanciasConsejero}.
\end{art}

\begin{art}\label{eleccionesConsejeriaVotacion}
    La elección se realizará en conjunto con la de la Directiva, por lo cual deberán seguir la condiciones establecidas en la sección \ref{votaciones}.
\end{art}

\subsection{De la elección de los Delegados}\label{eleccionesDelegados}

\begin{art}\label{eleccionesDelegadosFecha}
    La elección de Delegados será la segunda semana de abril, y no durará más de cinco días hábiles. La fecha exacta la dispondrá la Directiva al mando.
\end{art}

\begin{art}\label{eleccionesDelegadosCandidatos}
    Todo estudiante representado podrá postularse a ser Delegado del grupo que le corresponda, dentro de los mencionados en el artículo \ref{organismosDirectivos} punto iii. Las votaciones se llevarán a cabo como se describe en la subsección \ref{votacionesDelegados}.
\end{art}

\begin{art}\label{eleccionesDelegadosFalla}
    Si en alguna generación no existe, o se rechazan todas las postulaciones, el puesto quedará vacante, y se seguirá lo descrito en el artículo \ref{vacanciasConsejoDelegados}.
\end{art}


\subsection{De la elección del TRICEL}

\begin{art}\label{eleccionesTRICELFecha}
    La elección del TRICEL será la segunda semana de abril, junto con la elección de delegados.
\end{art}

\begin{art}\label{eleccionesTRICELCandidatos}
    Todo estudiante representado que no ocupe cargos dentro del CAM, podrá postularse a ser miembro del TRICEL. Las votaciones se llevarán acabo como se describe en la subsección \ref{votacionesTRICEL}
\end{art}

\section{Del Tribunal Calificador de Elecciones (TRICEL)}\label{TRICEL}

\begin{art}\label{TRICELMision}
    El TRICEL es el organismo encargado de organizar, vigilar y concretar todo proceso de votación a nivel CAM.
\end{art}

\begin{art}\label{TRICELDescripcion}
    El TRICEL estará compuesto por un mínimo de dos estudiantes representados, los cuales no tendrán algún otro cargo dentro del CAM.
\end{art}

\begin{art}\label{TRICELFunciones}
    Serán funciones del TRICEL:
    \begin{enumerate}
        \item Velar por la realización y garantizar la transparencia del proceso de votaciones.
        \item Conocer cualquier asunto relacionado con las votaciones que fiscalice.
        \item Garantizar el derecho a voto de los estudiantes representados.
        \item Calificar las votaciones dando su dictamen justificado respecto de la legitimidad o nulidad, de naturaleza parcial o total, del proceso.
        \item Determinar la lista de personas con derecho a voto de acuerdo a lo estipulado en el presente estatuto.
        \item Determinar si el universo de votantes de una votación incluye a los estudiantes representados de postgrado, previa consulta con el delegado de postgrado.
        \item Atender, investigar y resolver los reclamos y observaciones presentadas con respecto al proceso de votación por cualquier estudiante representado.
        \item Determinar, publicar, solicitar y distribuir el material necesario para la implementación de las votaciones.
        \item Garantizar la presencia de personal idóneo para el buen desempeño del proceso electoral. Para ello, el TRICEL podrá seleccionar a quienes estime convenientes para que preste ayuda en el mencionado proceso. Dicho personal no podrá estar integrado por ningún miembro que ejerza o sea candidato a alguno de los cargos a los que hace referencia el artículo \ref{organismosDirectivos} de los presentes estatutos.
    \end{enumerate}
\end{art}

\begin{art}\label{TRICELResolucion}
    Las resoluciones del TRICEL serán tomadas por acuerdo unánime y solo serán apelables ante el mismo tribunal por vía de reconsideración impulsada por no menos de cinco y no más de quince estudiantes representados que hayan votado. Luego, el TRICEL pasa a acordar decisión nueva. Esta segunda decisión será inapelable.
\end{art}

\section{Destituciones, Renuncias y Vacancias}\label{vacancias}

\begin{art}\label{perdidaCargosCAM}
    Todo cargo del CAM se mantiene hasta que se designe una nueva persona en el cargo correspondiente, y se pierde si se cumple alguna de las siguientes alternativas, o si se incumple alguna de las condiciones del artículo \ref{eleccionesRequerimientosPostulación}:
    \begin{enumerate}
        \item Renuncia voluntaria.
        \item Destitución por medios definidos en la sección \ref{destituciones}.
        \item Abandono de la carrera, en caso de ser Delegado o Consejero Académico.
        \item Suspensión de la carrera de forma voluntaria y temporal.
    \end{enumerate}
\end{art}

\subsection{Destituciones}\label{destituciones}

\begin{art}\label{destitucionesConsejo}
    El Consejo Estudiantil puede destituir miembros del CAM, con excepción del Consejero Territorial, si es que este así lo decide por una votación interna no secreta. Esto será comunicado por todos los medios oficiales del CAM, e incluirá las razones de esta destitución y la fecha de la misma.
\end{art}

\begin{art}\label{destitucionesAsamblea}
    En una Asamblea General se puede proponer la moción de destituir a algún representante del CAM, que debe ser aprobada bajo lo establecido en la sección \ref{votacionesGeneral}.
\end{art}


\subsection{Vacancias y Renuncias}\label{vacanciasRenuncias}

\begin{art}\label{renunciasGeneral}
    Cada integrante del CAM tiene la facultad de renunciar en caso de que así lo desee, pero esta facultad no puede ser usada de forma simultánea por más de un miembro, excepto en caso de renuncia de la Directiva completa.
\end{art}

\begin{art}\label{renunciasDirectivaMiembro}
    Al renunciar un integrante de la Directiva, esta tiene la facultad de proponer un posible reemplazante, el cual tendrá que ser aprobado por el Consejo Estudiantil.
\end{art}

\begin{art}\label{vacanciasDirectivaMiembro}
    En caso de que un integrante de la Directiva no proponga un reemplazante, o que el reemplazo no sea aprobado, este puesto será reemplazado de la siguiente forma:
    \begin{enumerate}
        \item Si el cargo es ejecutivo (i.e. Presidencia, Vicepresidencia Interna, Vicepresidencia Externa) este será reemplazado por el siguiente cargo ejecutivo en la cadena de mando, la cual por defecto será, en orden, Presidencia, Vicepresidencia Interna y por último Vicepresidencia Externa.
        \item Si el cargo es administrativo (i.e. Secretaría, Tesorería) este será reemplazado por el siguiente cargo administrativo en la cadena de mando, la cual por defecto será, en orden, Secretaría y por último Tesorería.
        \item En caso de que el último cargo en la cadena de mando respectiva quede vacío se sigue en orden alfabético los puntos descritos a continuación. Además, en todos los casos debe ser aprobado por el Consejo Estudiantil con una votación de mayoría simple:
              \begin{enumerate}[label=(\alph*)]
                  \item La Directiva puede proponer un miembro del Consejo Ejecutivo.
                  \item La Directiva puede proponer un miembro del Consejo Estudiantil.
                  \item El Consejo Estudiantil puede proponer un miembro del Consejo Ejecutivo.
                  \item El Consejo Estudiantil puede proponer un miembro del Consejo Estudiantil.
              \end{enumerate}
    \end{enumerate}
\end{art}

\begin{art}\label{vacanciasDirectivaPlazos}
    La renuncia de un integrante de la directiva debe ser presentada con una semana de anticipación, excepto en el caso de la Tesorería donde esta necesitará un plazo extra de dos semanas, para facilitar el proceso de cambio administrativo.
\end{art}

\begin{art}\label{vacanciasDirectivaCompleta}
    En caso de quedar vacante la posición de toda la Directiva asumirá el Consejo Estudiantil en forma de una Directiva interina\footnote{Ver artículo \ref{eleccionesDirectivaInscripcion}}, y se llamará a elecciones inmediatamente después de la renuncia. Si la renuncia ocurre durante el primer semestre, la Directiva electa terminará el período en curso, en caso contrario, la Directiva electa asume el período en curso y además un período completo.
\end{art}

\begin{art}\label{vacanciasConsejero}
    En caso de quedar vacante la posición del Consejero Académico, el cargo lo asume el Subconsejero Académico, y en caso de la ausencia de éste, lo asume quien tenga el cargo de la Vicepresidencia Interna, quedando con ambos cargos.
\end{art}

\begin{art}\label{vacanciasConsejoDelegados}
    En caso de quedar vacante la posición de un integrante del Consejo de Delegados, se llamará a elecciones abiertas bajo las mismas reglas expuestas en la subsección \ref{votacionesDelegados}, con la siguiente excepción: si no hay candidato, entonces el Consejo Estudiantil designará por votación interna a uno de sus miembros para que asuma este cargo; luego, se hará una votación aprobatoria según lo establecido en la subsección \ref{votacionesDelegados}. En caso de que el candidato no sea aprobado, o que no se haya llegado a un acuerdo en el Consejo Estudiantil, el puesto queda vacío.
\end{art}

\begin{art}\label{vacanciasConsejoEstudiantil}
    En caso de quedar vacante la posición del Director, o el Secretario, del Consejo Estudiantil se volverán a elegir estos cargos después de finalizar los procesos de los artículos \ref{renunciasDirectivaMiembro}, \ref{vacanciasDirectivaMiembro}, \ref{vacanciasDirectivaCompleta}, \ref{vacanciasConsejero} y \ref{vacanciasConsejoDelegados}.
\end{art}

\section{Votaciones}\label{votaciones}

\subsection{Votaciones Generales}\label{votacionesGeneral}

\begin{art}\label{}
    Toda votación tiene un universo de votantes, el cual por defecto será los estudiantes representados.
\end{art}

\begin{art}\label{}
    Toda votación tiene un quórum mínimo para ser válida, el cual por defecto será de 40\%.
\end{art}

\begin{art}\label{}
    Toda votación tiene duración, la cual por defecto será dos días hábiles.
\end{art}

\begin{art}\label{}
    Toda votación para una elección tiene un sistema electoral para determinar los individuos que son elegidos en los cargos correspondientes. Por defecto, el sistema es de mayoría simple.
\end{art}

\begin{art}\label{}
    Toda votación para una elección tiene un mínimo de aprobación necesaria para que los candidatos puedan llegar a asumir el cargo correspondiente. Por defecto, este mínimo será 40\%.
\end{art}

\begin{art}\label{}
    Toda votación para una elección donde haya una postulación única, será de Apruebo o Rechazo por defecto.
\end{art}

\subsection{Votaciones Consejería Académica}\label{votacionesConsejeriaAcademica}

\begin{art}\label{}
    En la votación para la elección de la Consejería Académica se tienen tres universos de votantes separados, compuestos por estudiantes representados de la carrera de Matemáticas, de la carrera de Estadística y de la carrera de Ciencias de Datos, correspondientemente. Para cada universo de votantes se harán votaciones, las cuales serán llevadas a cabo en paralelo.
\end{art}

\subsection{Votaciones Delegados}\label{votacionesDelegados}

\begin{art}\label{}
    En la votación para la elección de Delegados de cada generación, de postgrado y de los estudiantes representados no considerados en los grupos anteriores (ver artículo \ref{organismosDirectivos}, iii) son universos de votantes distintos y separados, por ende para cada universo de votantes se harán votaciones, las cuales serán llevadas a cabo en paralelo.
\end{art}

\subsection{Votaciones TRICEL}\label{votacionesTRICEL}

\begin{art}\label{}
    El sistema electoral para la elección de los miembros del TRICEL se describe a continuación.
    \begin{enumerate}
        \item Se votará individualmente por cada candidato donde las opciones serán Apruebo o Rechazo. Luego, se escogerán, a lo más, los cuatro candidatos que hayan obtenido mayor aprobación de los que hayan obtenido el mínimo de aprobación.
        \item En el caso de que las primeras cuatro mayorías sean de la misma carrera y exista un candidato con el mínimo de aprobación de alguna de las otras carreras, entonces se elegirá la primera mayoría de estos últimos como quinto miembro.
    \end{enumerate}
\end{art}

\subsection{Del Plebiscito}\label{plebiscito}

\begin{art}\label{plebiscitoDescripcion}
    El plebiscito es la consulta directa a todos los estudiantes, realizada mediante una votación, sobre materias especificas y que tiene carácter vinculante al interior del CAM.
\end{art}

\begin{art}\label{plebiscitoConvocar}
    Pueden convocar a Plebiscito:
    \begin{enumerate}
        \item La Directiva por unanimidad.
        \item La Asamblea General con mayoría absoluta de sus votos.
        \item El Consejo Estudiantil con mayoría absoluta de sus votos.
    \end{enumerate}
\end{art}

\begin{art}\label{plebiscitoDocumento}
    En caso de que la Asamblea General convoque un plebiscito, se creará una comisión la cual estará encargada de crear un documento con las alternativas de forma clara, precisa y completa. En caso de que un organismo del CAM convoque un plebiscito, este tomará el lugar de la comisión.
\end{art}

\begin{art}\label{plebiscitoAntelacion}
    La convocatoria a plebiscito, junto al documento con las alternativas, deberá ser presentada con al menos cinco días hábiles de antelación al plebiscito.
\end{art}

\begin{art}\label{}
    En caso de que el plebiscito no sea válido, este no tendrá carácter vinculante.
\end{art}

\section{Vocalías}\label{vocalias}
\begin{art}\label{}
    Las vocalías son organismos independientes creadas con un enfoque específico.
\end{art}

\begin{art}\label{}
    Cada vocalía deberá estar conformada por un mínimo de dos estudiantes representados y deberá ser aprobada por el Consejo Estudiantil.
\end{art}

\begin{art}\label{}
    Como organismo independiente, estas pueden conseguir financiamiento propio y/o solicitar apoyo financiero formalmente a la Directiva.
\end{art}

\begin{art}\label{vocaliasSolicitud}
    En caso de acudir a la Directiva por apoyo financiero, debe enviarse una solicitud escrita que incluya:
    \begin{enumerate}
        \item Proyecto asociado.
        \item Monto a solicitar.
        \item Metodología de implementación.
        \item Firmas de todos los integrantes.
    \end{enumerate}
    La solicitud tiene que ser aprobada por el Consejo Estudiantil y la Directiva.
\end{art}

\begin{art}\label{}
    Si la solicitud especificada en el artículo \ref{vocaliasSolicitud} es aprobada, se deberá presentar respaldo del uso de los fondos cada tres meses desde su adjudicación, o cuando se finalice el proyecto asociado.
\end{art}

\section{De los Estatutos}\label{estatutos}

\begin{art}\label{estatutosTipoReformas}
    Existen dos tipos de propuestas a reforma:
    \begin{enumerate}
        \item Propuesta de cambio gramatical.
        \item Propuesta de cambio estructural.
    \end{enumerate}
\end{art}

\begin{art}\label{estatutosCambioGramatical}
    Todo cambio gramatical debe ser discutido y aprobado en el Consejo Estudiantil.
\end{art}


\begin{art}\label{}
    Podrán llamar a cambio estructural de los presentes estatutos:
    \begin{enumerate}
        \item La Asamblea General, con mayoría simple.
        \item El Consejo Estudiantil, con dos tercios de los votos.
        \item Miembros de la Directiva, por unanimidad.
    \end{enumerate}
\end{art}

\begin{art}\label{estatutosComision}
    Este llamado a cambio debe ser presentado en la Asamblea General, acompañado de argumentación que justifique la solicitud. Junto a esto, se debe hacer un llamado abierto, por todos los medios oficiales del CAM, para conformar una comisión. Esta realizará una propuesta de reforma a los presentes estatutos, la cual debe seguir los lineamientos del llamado a cambio antes mencionado.
\end{art}

\begin{art}\label{estatutosReformaAprobacion}
    Una vez la comisión antes mencionada tenga lista una propuesta a reforma, esta se deberá  presentar por escrito a la Asamblea General, explicando los cambios que se proponen, junto con su argumentación respectiva. Posteriormente, esta se llevará a votación siguiendo lo detallado en la sección \ref{votaciones}.
\end{art}

\begin{art}\label{estatutosVigenciaReformas}
    Toda reforma aprobada entrará en vigencia una vez publicada, a menos que se establezca un período de vacancia.
\end{art}

\begin{art}\label{estatutosArticulosTransitorios}
    Si una reforma genera una incompatibilidad con los estatutos vigentes, esta deberá contener artículos transitorios que establezcan soluciones a estas incompatibilidades. La vigencia de estos no puede extenderse más de un año de su publicación y, terminado este plazo, automáticamente son eliminados de los estatutos vigentes.
\end{art}

\begin{art}\label{}
    En caso de que se detecte una incompatibilidad entre normas de los presentes estatutos, esta incompatibilidad será presentada en Asamblea General para que esta se pronuncie sobre ella y entregue una solución, como se describe en el artículo \ref{estatutosInterpretacion}. Además, se creará una comisión según lo descrito en el artículo \ref{estatutosComision} con la finalidad de solucionar esta incompatibilidad.
\end{art}

\begin{art}\label{}
    Se deberá mantener un historial de cambios y propuestas, tanto aprobadas como rechazadas, de los estatutos en la plataforma GitHub (\url{https://github.com/CAM-UC/Estatutos}). La administración de este historial será responsabilidad de quien tenga el cargo de la Secretaría, o de un estudiante representado aprobado por el Consejo Estudiantil, el cual deberá usar el sistema de ``pull requests'' y ``branches'' para mantener de manera ordenada los cambios realizados y propuestos, detallándolos de la mejor manera posible.
\end{art}

\begin{art}\label{estatutosInterpretacion}
    En caso de existir cualquier duda sobre la interpretación del presente estatuto, el Consejo Estudiantil deberá presentar una interpretación. Esta será discutida y podrá ser modificada en la Asamblea General, para posteriormente ser sometida a votación en la misma Asamblea.
\end{art}

% TODO Considerar convenciones en la documentación y los cambios de los estatutos para facilitar las futuras reformas y la documentación de los cambios.

\section{Disposiciones Transitorias}\label{transitorias}

\begin{artt}\label{}
    La Consejería Académica de la carrera de Ciencias de Datos se constituará solo si existen al menos dos generaciones de la misma. Ambas Consejerías (Matemática y Estadística) suplirán este cargo hasta que se constituya. Esta disposición podrá ser renovada una vez por ratificación del Consejo Estudiantil.
\end{artt}
% * La siguiente parte corresponde a la documentación de futuras y la presente reforma.
% * Todo equipo correspondiente a una reforma deberá añadirse abajo incluyendo la información pertinente (e.g. mes, año, miembros del equipo en cuestión)
\newpage
\textsc{Elaborado por Comisión de reforma de estatutos:}
Elías Alvear, Sebastián Avendaño, Matías Bruna, María Catalina Cárdenas, Ángela Flores, Nicholas Mc-Donnell, Romina Mercado, José Antonio Montenegro, Javier Reyes, Juan Pablo Vega y Paulina Vega.
\begin{sloppypar}
    Facultad de Matemáticas, Pontificia Universidad Católica de Chile\\
    Santiago, Julio de 2020
\end{sloppypar}
\ \\
\textsc{Modificados por Comisión para Data Science:}
Camila Guajardo, Nicholas Mc-Donnell, José Antonio Montenegro, Bernardo Mundaca, Javier Reyes y Juan Pablo Vega
\begin{sloppypar}
    Facultad de Matemáticas, Pontificia Universidad Católica de Chile\\
    Santiago, Diciembre 2020
\end{sloppypar}
\end{document}
